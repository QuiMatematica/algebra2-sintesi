\chapter{Cuore normale}
\label{ch:cuore_normale}

\section{Definizione}
\label{sec:cuore_normale_definizione}

\begin{equation*}
    H_G = \bigcap_{x \in G} xHx^{-1} \quad \text{con } H \le G
\end{equation*}

\section{Proprietà}
\label{sec:cuore_normale_proprieta}

\begin{itemize}
    \item $H_G \subseteq H$
    \item $H_G \normale G$
    \item $H_G$ è il più grande sottogruppo normale di $G$ contenuto in $H$.

        Ogni altro sottogruppo normale di $G$ contenuto in $H$ è contenuto in $H_G$.
    \item $\indice{G}{H} = m \allora \indice{G}{H_G} \dividetxt m!$
\end{itemize}

\section{Applicazioni}
\label{sec:cuore_normale_applicazioni}

$A_5$ contiene sottogruppi di ordine 15 (e indice 4)?

Poniamo per assurdo che esista un sottogruppo $H \le A_5$ di ordine 15 e indice 4.
\begin{equation*}
    \indice{A_5}{H} = 4 \allora \indice{A_5}{H_{A_5}} \dividetxt 4! = 24
\end{equation*}

$A_5$ è un gruppo semplice, quindi $H_G$ non può che essere il sottogruppo identico:
\begin{equation*}
    H_{A_5} = 1 \allora \indice{A_5}{H_{A_5}} = \ordine{A_5} = 60
\end{equation*}

60 non divide 24, quindi l'ipotesi per assurdo è falsa e non esiste un sottogruppo di $A_5$ di ordine 15.



